Such data, says Mr Macdonald, are attractive to teams looking for an edge over
the competition. Previous efforts have relied on the Global Positioning System
of satellites, which offers much lower accuracy. The firm has tested its
technology with several professional clubs\cite{chen2017computation,
al-shuwaili2017energyefficient}, including Saracens, the reigning champions in
the English Premiership\cite{ding2019beef}. Where the fun starts, though, is
when similar sensors are put into the ball. It can then, metaphorically, squawk
if passed forward (which is illegal in rugby), and there will be no doubt, by
comparing the positions of ball and player, when a player is offside. A smart
ball will be able to monitor other rules, too. It was, for instance, tested
successfully in a five-a-side version of the game called Rugby X, in which you
are not allowed to kick the ball higher than ten meters. Since few referees are
equipped with theodolites, enforcing this rule has been hard\cite{fan2017cost,
chantre2018multiobjective, santoyo-gonzalez2018edge,wang2018edge ,
zhao2018optimala, jia2017optimal, meng2017cloudlet ,xu2016efficient, %abcd
chen2017computation}. Now it is easy.